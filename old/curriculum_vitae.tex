\documentclass[a4paper]{article}
\usepackage{tabularx}

%\usepackage{doublespace}
%\setstretch{1.2}

\usepackage{ae}
\usepackage{a4wide}
\usepackage[T1]{fontenc}
\usepackage{curriculum_vitae}
\usepackage{hyperref}
%\usepackage{garamon}
\usepackage{rcs}

\RCS $Id$
\RCS $Date$
\RCS $Revision$

% http://linuxgazette.net/issue20/latex.html
\usepackage{fancyhdr}
\renewcommand{\headrulewidth}{0.5pt}
\renewcommand{\footrulewidth}{0pt}
\fancypagestyle{rcsfooters}{%
\fancyhf{}
\fancyhead{} % get rid of headers
\renewcommand{\headrulewidth}{0pt} % and the top line
\fancyfoot[L]{}
\fancyhead[C]{}
\fancyfoot[R]{Revision: \RCSRevision}}

\begin{document}
\pagestyle{empty}
\begin{center}
\Huge{\textsc{Curriculum Vit\ae}}
\vspace{\baselineskip}

\Large{\textsc{Joachim Nilsson}}
\end{center}
\vspace{1.5\baselineskip}

\section{Personal Data}
\begin{personals}
  \info{Name:}
        {Lars \underline{Joachim} Nilsson}
  \info{Born:}
        {July 23rd, 1974, Nacka, Stockholm}
  \info{Address:}
        {\begin{minipage}[t]{8cm}
        Morkullegatan~38      \\
        S-724~69~V�STER�S     \\
        SWEDEN
        \end{minipage}}
  \info{Tel:}
        {+46(0)21~--~12~33~48}
  \info{Civil status:}
        {Married, two kids}
  \info{Email:}
        {\url{joachim.nilsson@vmlinux.org}}
  \info{Web:}
        {\url{http://vmlinux.org/joachim/}}
  \info{Language skills:}
        {Fluent in English \& Swedish, with some limited understanding of French}
\end{personals}


\section{Work Experience}
\begin{CV}
  \row{2002--Present}
      {\textbf{Consultant, embedded and Linux systems, �F-System AB, V�ster�s} \\
\emph{2005: Tutor ....}
\emph{2004: ABB Force Measurement --- Network Security Analysis.}
Security analysis and firewall recommendations for connecting a time
critical industrial network to an office network with Internet access.

The Stressometer flatness measuring system is an advanced flatness
system for rolling mills with high demands on network load
predictability and quality of service.

\emph{2002--Now: EssNet AB --- Linux USB2 driver development.}
Development of several Linux kernel device drivers for a highly
advanced lottery system.  In particular a Cypress FX2 (USB 2.0) based
high--speed scanner with functions for scanning and calibration,
branding of printed receipts, cashdrawer and dedicated serial port
interface.  First devleoped for Linux kernel 2.4 and later ported
to Linux 2.6.

Responsible for continous maintenance of drivers and Linux system software.}

  \row{2000--2002}
      {\textbf{R\&D Engineer, RealFast Operating Systems AB, V�ster�s} \\
\emph{2002: Mentor Graphics Inc. --- Development project in Linux.}
A six month project for Mentor Graphics to port Linux to the RealFast
H/W microkernel, similar to the VxWorks project (below). Testbench (a
S/W simulator of the microkernel), complete system including drivers
redesign of the scheduler etc., all developed using GNU/Linux.

Debugging of the ARM Integrator platform with the Arm AxD debugger
using a MultiICE JTAG probe.

\emph{2001: M�lardalen University, V�ster�s.} Lecturer and examiner
for a 5p course in C programming at the Department of Computer
Engineering, IDt.  See the course page for details: \url{http://www.idt.mdh.se/kurser/cd5020/jnnht01/}

\emph{2001: Sierra S16.} Project leader and S/W developer for the
Sierra real-time kernel. A minimalistic OS based on the RealFast
H/W microkernel wrapped with a S/W API to the hardware coupled with
an adaptation of NewLib to provide a limited C library.

Used today in courses given at M�lardalen University.

\emph{2001: Ericsson Radio Systems AB, Nacka Strand.} Investigation
and demostration of how Linux and RTLinux can replace Enea OSE in
Ericsson switches based on the GPB2, General Purpose Board 2.

\emph{2001: Applied Linux \& Embedded Internet Show, Kista.}
Presented, on 5th April, Linux and other free kernels for embedded and
real-time systems. A rundown of the embedded OS's eCos and uClinux as
well as the real-time OS's RTLinux and RTAI. Disscussed how each
could be applied to a certain set of problems and what to whatch out
for when choosing a free operating system.

\emph{2000: Ericsson Mobile Data Design AB, Gothenburg.} A ten month
project with extensive modifications of VxWorks and its microkernel
Wind to support the H/W microkernel developed by RealFast.  The latter
is implemented in VHDL prototyped on a PMC card with an FPGA.

The goal of the project was to compare the performance of VxWorks
with and without the hardware acceleration. This was done with a
simple IP forwarding application running in VxWorks on the Ericsson
GIC (General Interface Carrier) board.

The project included debugging of IBM PowerPC 750 using IBM RISCWatch
with a JTAG probe.  Including debugging of PCI drivers with SingleStep
and the Vmetro PMC bus analyzer.

Tests and performance measurements made for the evaluation of the
project was, amongst other tools, carried out with an advanced
IP packet generator.

The project also included modification and auditing of Ericsson
developed drivers and base platform in VxWorks.

\emph{2000--2002: RealFast (internal work).} Network and systems
administration of Linux, OpenBSD and Solaris machines.

Ported VxWorks to the Real-Time Unit, RTU, a real-time kernel
implemented in VHDL. All kernel primitives in VxWorks was adapted to
use the H/W micro kernel.}

  \row{1998--1999}
        {\textbf{Software developer, ABB Network Partner AB, V�ster�s} \\
Developed a configuration compiler using Bison \& Flex.  An older DOS
based configuration program for high--currency relay protection systems
was to be ported to Windows.  The compiler was used to parse
configuration files and constructed as quite an advanced compiler
using Bison and Flex, with a lot of C glue code.

The compiler was fed protocol and configuration data and then
generated an MS~Access database over an ODBC bridge.  Everything
developed using GNU software, Emacs and Cygwin, on Windows NT.}

  \row{1996, 1997}
      {\textbf{Summer vacation temp., service engineer for medical devices} \\
Service and maintenance of radiology equipment at Kullbergska hospital
in Katrineholm and Nyk�pings lasarett.}

  \row{1995}
      {\textbf{Service engineer for medical devices} \\
Worked as a service and maintenance engineer of medical equipment,
mostly for radiology, at Kullbergska sjukhuset, Katrineholm.
Responsible for regular maintenace and repairs of radiology equipment,
including X--ray machines, film development machines, etc.
Maintenance of other types of medical equipment was also part of the
job, e.g., ventilators and electric shock devices.}

%  \row{1989}
%        {Summer holiday work in production, Hemglass AB, Str�ngn�s}
%
%  \row{1987, 1988}
%        {Summer holiday work in production, Drinkit AB, Str�ngn�s}
\end{CV}

\section{References}
\noindent The following people are familiar with both my professional
qualifications and character:

\begin{CV}
  \row{Johan Risberg, EssNet project manager}
      {�F-System AB, +46--(0)8--657~16~18}

      Johan~Risberg~\url{<johan.risberg@afconsult.com>}

   \row{Lennart Lindh, Ph.D.,}
        {RealFast AB, +46--(0)70--668~95~17

        Lennart~Lindh~\url{<lennart.lindh@realfast.se>}}

   \row{Prof. Gerhard Fohler, Ph.D.,}
        {MRTC\footnote{M�lardalen Real--Time Research Centre, \url{<http://www.mrtc.mdh.se/>}}, V�ster�s, +46--(0)21~10~31~58

        Gerhard~Fohler~\url{<gerhard.fohler@mdh.se>}}

   \row{Mikael Bergqvist, Ph.D.,}
        {NetInsight AB, +46--(0)70--831~77~62

        Mikael~Bergqvist~\url{<mikael.bergqvist@netinsight.com>}}

   \row{Bengt Jervmo,}
        {BlueLabs AB (formerly Frontec Tekniksystem AB), +46--(0)8--470~22~19

        Bengt Jervmo \url{<bengt.jervmo@bluelabs.se>}}

   \row{Mikael Wilroth,}
	{ABB Network Partner AB, V�ster�s, +46--(0)21--34~20~00

	Mikael~Wilroth~\url{<mikael.wilroth@se.abb.com>}}

%   \row{Personnel manager,}
%        {Hemglass Sverige AB, Str�ngn�s, +46--(0)152--161~00.}
%
%   \row{Anne--Lie L��v,}
%        {Drinkit AB, Str�ngn�s, +46--(0)152--180~90.}
\end{CV}

\newpage
\section{Education}
\begin{CV}
  \row{1999--2000}
      {Masters Degree in Real-Time Systems, Dept. of Computer
        Engineering, University of M�lardalen, V�ster�s. Thesis:
        \emph{Modular Scheduling in RTLinux}, supervisors:
        Prof. Gerhard Fohler and Mikael Bergqvist, Frontec AB.}

  \row{February 2000}
        {Introduction to Tornado 2/VxWorks 5.4, Frontec AB.}

  \row{Spring 1999}
        {Separate courses in English, University of M�lardalen, V�ster�s.}

  \row{1995--1998}
        {Bachelor of Science in Computer Engineering.

        University of M�lardalen, V�ster�s.}

  \row{1994--1995}
        {Service engineer for medical devices, civil service.}

  \row{1993--1994}
        {Fourth year college engineer, specialised in ``Industrial
          electronics''.

        Rinmansskolan, Eskilstuna.}

  \row{1990--1993}
        {Three years of technical college. Final work in \emph{Nuclear
            Physics}.

        Thomasgymnasiet, Str�ngn�s.}

%  \row{1981--1990}
%        {Elementary school, Vasaskolan, Str�ngn�s}
%
\end{CV}

\newpage
\section{Thesis}
``Modular Scheduling in RTLinux''; a Master thesis work done in 2000
with Daniel~Rytterlund \url{<daniel.rytterlund@vmlinux.org>}. The task
was to examine existing real--time solutions for Linux, choosing the
most suitable,  from some basic criteria, and to implement a couple of
well known scheduling algorithms. The goal was to lay a foundation for
a completely modular scheduler capable of scheduling all types of
tasks; sporadics, aperiodics and periodics.

The following is the abstract from the thesis.

\subsection*{Abstract}
Traditionally, the only choice a software engineer had was
closed source commercially, or in-house, developed real-time operating
systems. Similarly, in the academic world several experimental
real-time operating systems have been developed.

During the course of the last decade a new operating system and a new
software development model has been gaining momentum. The Linux
operating system has paved the road for the Open Source model of
software development.

The freely available Linux operating system, combined with the expert
knowledge of real-time systems within academia, has recently given
birth to several Linux based real-time operating systems, some of
which have proven to be viable alternatives to the existing commercial
ones.

RTLinux, RTAI, KURT, and RED-Linux are the four real-time Linux based
operating systems which have been examined. RTLinux, originating from
the New Mexico Institute of Technology, is a hard RTOS characterised
by a stand-alone real-time kernel running the standard Linux kernel as
its lowest priority task. Similar in operation is RTAI, developed at
Politecnico di Milano, which in addition employs an abstraction layer
between the two kernels. As opposed to the two former hard real-time
operating systems, KURT as well as RED-Linux are soft RTOSs. In the
latter two cases the standard Linux kernel has been enhanced with
real-time capabilities.


Our contributions consists of an investigation of the four mentioned
RTOSs and an implementation of a modular scheduler for RTLinux. The
implementation is mainly concerned with the separation of the
scheduling policy from the scheduling module in RTLinux, making it a
separate interchangeable module. This simplifies the development of
various scheduling algorithms in RTLinux.

Included in the appendix are, not only a complete listing of our
modifications to the source code, but also a glossary of acronyms and
abbreviations. Moreover, we also provide an installation manual for
RTLinux as well as a cross compiler tutorial. The installation manual
covers topics such as retrieving necessary components, installing,
configuring the real-time kernel and running the bundled examples. The
cross compiler tutorial is a short description of how to build and use
a cross development environment for an embedded system.


\section{Civil service}
\begin{CV}
  \row{1994--1995}
        {Service engineer for medical devices.

        Pliktverket \url{<http://www.pliktverket.se/>}}
\end{CV}

%
%\begin{flushright}
%\vspace{2.5cm}
%\noindent V�ster�s, \RCSDate
%
%
%\vspace{1cm}
%\_\_\_\_\_\_\_\_\_\_\_\_\_\_\_\_\_\_\_\_\_\_\_\_\_\_\_\_\_\_\_\_\_\_\_\_\_\_\_\_\_\_\_\_\_\_\_\_\_\_\_\_\_\_\_\_\_\_
%
%\vspace{.3cm}
%\noindent Joachim Nilsson
%\end{flushright}
\thispagestyle{rcsfooters}
\end{document}
